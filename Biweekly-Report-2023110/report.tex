\documentclass[a4paper]{article}

\usepackage[english]{babel}
\usepackage[utf8]{inputenc}
\usepackage{fullpage}
\usepackage{amsmath}
\usepackage{graphicx}
\usepackage[colorinlistoftodos]{todonotes}
\usepackage{hyperref}
\usepackage{amssymb}
\usepackage{outline} \usepackage{pmgraph} \usepackage[normalem]{ulem}
\usepackage{graphicx} \usepackage{verbatim}
% \usepackage{minted} % need `-shell-escape' argument for local compile

\usepackage[UTF8]{ctex}
\usepackage[inkscapeformat=png]{svg}

\title{
    \vspace*{1.0in}
    \includesvg[width=2.75in]{figures/logo.svg} \\
    \vspace*{1in}
    \textbf{\Huge Biweekly Report}
    \vspace{0.5in}
}

\author{ \\
    \textbf{\huge userElaina} \\
    \vspace*{1in}
}

\date{\LARGE 十一月上}
\setcounter{page}{-1}
\newpage

\begin{document}
\LARGE

\maketitle
\tableofcontents
% \setcounter{page}{0}
\thispagestyle{empty}
\newpage

\section{参加 MLA 2023}

\href{https://www.lamda.nju.edu.cn/conf/mla23/index.html}{MLA 2023} (第二十一届中国机器学习及其应用研讨会) 于南京召开.

\section{组会交流}

\subsection{MLA 2023}

MLA 2023 结束后, 于组会进行心得分享.

组会 ppt: \href{https://github.com/mscelaina/ppt-mla23}{MLA 23}.

\subsection{优化基础}

准备 11 月 20 日 9:00 pm 的组会分享: 优化基础.

(ppt 未完成)

\subsection{面向人工智能研究的计算机技术}

准备 11 月底的组会分享: 面向人工智能研究的计算机技术分享.

(ppt 未完成)

\section{课程}

\subsection{政治}

考试和作业.

\subsection{数据知识融合}

完成 Project 及 Pre.

项目地址(Organization): \href{https://github.com/KDI-2023}{JLU Tag}.

\section{开源项目}

与其它 Contributors 讨论项目 \href{https://github.com/userElaina/Open-JLU}{Open-JLU} 的相关事宜.

其它个人项目维护.

\end{document}
